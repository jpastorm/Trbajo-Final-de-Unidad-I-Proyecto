\documentclass[twoside,twocolumn]{article}

\usepackage{blindtext} 
\usepackage{graphicx}
\usepackage[sc]{mathpazo} 
\usepackage[T1]{fontenc} 
\linespread{1.05} 
\usepackage{microtype} 


\usepackage[english]{babel} 


\usepackage[hmarginratio=1:1,top=32mm,columnsep=20pt]{geometry} 
\usepackage[hang, small,labelfont=bf,up,textfont=it,up]{caption} 
\usepackage{booktabs} 


\usepackage{lettrine} 


\usepackage{enumitem} 
\setlist[itemize]{noitemsep} 


\usepackage{abstract} 
\renewcommand{\abstractnamefont}{\normalfont\bfseries} 
\renewcommand{\abstracttextfont}{\normalfont\small\itshape} 


\usepackage{titlesec} 
\renewcommand\thesection{\Roman{section}} % 
\renewcommand\thesubsection{\roman{subsection}} 
\titleformat{\section}[block]{\large\scshape\centering}{\thesection.}{1em}{} 
\titleformat{\subsection}[block]{\large}{\thesubsection.}{1em}{} 


\usepackage{fancyhdr} 
\pagestyle{fancy} 
\fancyhead{} 
\fancyfoot{} 
\fancyhead[C]{Titulo $\bullet$ Junio 2019 $\bullet$ } 
\fancyfoot[RO,LE]{\thepage} 


\usepackage{titling} 


\usepackage{hyperref} 


%----------------------------------------------------------------------------------------
%	TILULOS
%----------------------------------------------------------------------------------------


\setlength{\droptitle}{-4\baselineskip} 

\pretitle{\begin{center}\Huge\bfseries} 
\posttitle{\end{center}} 
\title{Business Intelligence and Business Analytics} 
\author{Andre Reinoso Aranda, Marko Antonio Rivas Rios, Andree Velasco Sucapuca, \\
Percy Taquila Carazas, Roberto Zegarra Reyes. }
\date{\today} 
\renewcommand{\maketitlehookd}{
\begin{abstract}
\noindent 
Business Intelligence BI is a tool, below different kind organizations, supports
decisions making processes, based in an exact and accurate information;
guarantying the production of the needed knowledge that lets to choose the most
appropiate option for the company success. The investigation begins with the BI
definition and applications; by addition shows definitions and relevant BI
investigations tools, like Data Warehouse, Olap, Balance Scorecard and Data
Mining.
\end{abstract}
\begin{abstract}
\noindent 

La Inteligencia de Negocios BI (Business Intelligence) es una herramienta bajo
la cual diferentes tipos de organizaciones, pueden soportar la toma de decisiones
basadas en información precisa y oportuna; garantizando la generación del
conocimiento necesario que permita escoger la alternativa que sea más
conveniente para el éxito de la empresa. La investigación comienza con la
definición y aplicaciones de BI; además se muestran trabajos relevantes en
algunas de las herramientas para hacer BI, como son Data Warehouse (Bodega
de Datos), Olap (Cubos Procesamiento Analítico en Línea), Balance Scorecard
(Cuadro de Mando) y Data Mining (Minería de Datos). 

\end{abstract}
}

%----------------------------------------------------------------------------------------

\begin{document}

% Print the title
\maketitle

%----------------------------------------------------------------------------------------
%	INTRODUCCION
%----------------------------------------------------------------------------------------

\section{Introduccion}
\lettrine[nindent=0em,lines=3]{A}ctualmente en el Perú las pequeñas y medianas empresas producen al mercado peruano ingresos y empleo, la gran cantidad de informacion que manejan es debido al alto numero de operaciones que realizan a diario.\\ \\

\section{Titulo}
Actualmente en el Perú las pequeñas y medianas empresas producen al mercado peruano ingresos y empleo, la gran cantidad de informacion que manejan es debido al alto numero de operaciones que realizan a diario.\\ \\

\section{Autores}
Actualmente en el Perú las pequeñas y medianas empresas producen al mercado peruano ingresos y empleo, la gran cantidad de informacion que manejan es debido al alto numero de operaciones que realizan a diario.\\ \\

\section{Planteamiento del problema}
Actualmente en el Perú las pequeñas y medianas empresas producen al mercado peruano ingresos y empleo, la gran cantidad de informacion que manejan es debido al alto numero de operaciones que realizan a diario.\\ \\

\subsection{Problema}
Actualmente en el Perú las pequeñas y medianas empresas producen al mercado peruano ingresos y empleo, la gran cantidad de informacion que manejan es debido al alto numero de operaciones que realizan a diario.\\ \\

\subsection{Justificacion}
Actualmente en el Perú las pequeñas y medianas empresas producen al mercado peruano ingresos y empleo, la gran cantidad de informacion que manejan es debido al alto numero de operaciones que realizan a diario.\\ \\

\subsection{Alcance}
Actualmente en el Perú las pequeñas y medianas empresas producen al mercado peruano ingresos y empleo, la gran cantidad de informacion que manejan es debido al alto numero de operaciones que realizan a diario.\\ \\

\section{Objetivos}
Actualmente en el Perú las pequeñas y medianas empresas producen al mercado peruano ingresos y empleo, la gran cantidad de informacion que manejan es debido al alto numero de operaciones que realizan a diario.\\ \\

\subsection{General}
Actualmente en el Perú las pequeñas y medianas empresas producen al mercado peruano ingresos y empleo, la gran cantidad de informacion que manejan es debido al alto numero de operaciones que realizan a diario.\\ \\

\subsection{Especificos}
Actualmente en el Perú las pequeñas y medianas empresas producen al mercado peruano ingresos y empleo, la gran cantidad de informacion que manejan es debido al alto numero de operaciones que realizan a diario.\\ \\

\section{Referentes teoricos}
Actualmente en el Perú las pequeñas y medianas empresas producen al mercado peruano ingresos y empleo, la gran cantidad de informacion que manejan es debido al alto numero de operaciones que realizan a diario.\\ \\

\section{Desarrollo de la propuesta}
Actualmente en el Perú las pequeñas y medianas empresas producen al mercado peruano ingresos y empleo, la gran cantidad de informacion que manejan es debido al alto numero de operaciones que realizan a diario.\\ \\

\subsection{Tecnologia de informacion}
Actualmente en el Perú las pequeñas y medianas empresas producen al mercado peruano ingresos y empleo, la gran cantidad de informacion que manejan es debido al alto numero de operaciones que realizan a diario.\\ \\

\subsection{Metodologia, tecnicas usadas}
Actualmente en el Perú las pequeñas y medianas empresas producen al mercado peruano ingresos y empleo, la gran cantidad de informacion que manejan es debido al alto numero de operaciones que realizan a diario.\\ \\

\section{Cronograma}
Actualmente en el Perú las pequeñas y medianas empresas producen al mercado peruano ingresos y empleo, la gran cantidad de informacion que manejan es debido al alto numero de operaciones que realizan a diario.\\ \\


%----------------------------------------------------------------------------------------
%	BIBLIOGRAFIA
%----------------------------------------------------------------------------------------


\begin{thebibliography}{99} 

\bibitem[Silvia Chavez y Carmen Contreras, 2018]{}
\newblock Implementación de Business Intelligence, para el proceso de toma de decisiones del área de ventas.

\bibitem[Hans Peter Luhn 1958]{}
\newblock A Business Intelligence System

\bibitem[Alex Rayón, 2015]{Universidad de Deusto}
Conceptos básicos del Business Intelligence.

\bibitem[Jordi Conesa y Josep Curto, 2010]{}
\newblock Introduccion al Business Intelligence

\bibitem[Margaret Rouse, 2019]{}
\newblock Análisis de negocios (BA)

\bibitem[Noodle Editorial Staff, 2018]{}
\newblock Business analytics career paths

\bibitem[Josep Lluis Cano, 2007]{}
\newblock Business Intelligence: competir con información


\bibitem[Curto J., 2010]{} 
\newblock Introducción al Business Intelligence. Editorial UOC.

\bibitem[Kimball R. and Ross M., 2002]{} 
\newblock  The Data Warehouse Toolkit: The Complete Guide to Dimensional Modeling. Wiley
 
\end{thebibliography}


%----------------------------------------------------------------------------------------


\end{document}
